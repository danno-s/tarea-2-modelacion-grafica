% Template:     Informe/Reporte LaTeX
% Documento:    Archivo principal
% Versión:      3.1.3 (17/04/2017)
% Codificación: UTF-8
%
% Autor: Pablo Pizarro R.
%        Facultad de Ciencias Físicas y Matemáticas.
%        Universidad de Chile.
%        pablo.pizarro@ing.uchile.cl, ppizarror.com
%
% Sitio web del proyecto: [http://ppizarror.com/Template-Informe/]
% Licencia: MIT           [https://opensource.org/licenses/MIT]

% CREACIÓN DEL DOCUMENTO, FUENTE E IDIOMA
\documentclass[letterpaper,11pt]{article} % Articulo tamaño carta, fuente 11
\usepackage[utf8]{inputenc}               % Codificación UTF-8
\usepackage[T1]{fontenc}                  % Soporta caracteres acentuados
\usepackage{lmodern}                      % Tipografía moderna
\usepackage[spanish]{babel}               % Idioma del documento en español
\def\templateversion{3.1.3}               % Versión del template
               
% INFORMACIÓN DEL DOCUMENTO
\newcommand{\nombredelinforme}{Título del informe}
\newcommand{\temaatratar}{Tema a tratar}
\newcommand{\fecharealizacion}{\today}
\newcommand{\fechaentrega}{14 de mayo de 2017}

\newcommand{\autordeldocumento}{Nombre del autor o grupo}
\newcommand{\nombredelcurso}{Curso}
\newcommand{\codigodelcurso}{CO-1234}

\newcommand{\nombreuniversidad}{Universidad de Chile}
\newcommand{\nombrefacultad}{Facultad de Ciencias Físicas y Matemáticas}
\newcommand{\departamentouniversidad}{Departamento de la Universidad}
\newcommand{\imagendeldepartamento}{images/departamentos/dcc}
\newcommand{\imagendeldepartamentoescl}{0.2}
\newcommand{\localizacionuniversidad}{Santiago, Chile}

% INTEGRANTES, PROFESORES Y FECHAS
\newcommand{\tablaintegrantes}{
\begin{minipage}{1.0\textwidth}
\begin{flushright}
\begin{tabular}{ll}
	Integrante:
		& \begin{tabular}[t]{@{}l@{}}
			Daniel Soto G.
		\end{tabular} \\
	Profesora:
		& \begin{tabular}[t]{@{}l@{}}
			Nancy Hitschfeld K.
		\end{tabular} \\
	Auxiliares:
		& \begin{tabular}[t]{@{}l@{}}
			Pablo Pizarro R. \\
			Pablo Polanco
		\end{tabular}\\
	Ayudantes:
		& \begin{tabular}[t]{@{}l@{}}
			Joaquín T. Paris \\
			Rodrigo E. Ramos T. \\
			Sergio Leiva
		\end{tabular}\\
	\multicolumn{2}{l}{Fecha de realización: \fecharealizacion} \\
	\multicolumn{2}{l}{Fecha de entrega: \fechaentrega} \\
	\multicolumn{2}{l}{\localizacionuniversidad}
\end{tabular}
\end{flushright}
\end{minipage}}

% CONFIGURACIONES
\input{lib/config}

% IMPORTACIÓN DE LIBRERÍAS
\input{lib/imports}

% IMPORTACIÓN DE FUNCIONES
\input{lib/functions}

% IMPORTACIÓN DE AMBIENTES Y ESTILOS
\input{lib/styles}

% CONFIGURACIÓN INICIAL DEL DOCUMENTO
\input{lib/initconf}

% PORTADA
\begin{document}
\input{lib/portrait}

% CONFIGURACIÓN DE PÁGINA Y ENCABEZADOS
\input{lib/pageconf}

% RESUMEN O ABSTRACT
% Template:     Informe/Reporte LaTeX
% Documento:    Abstract o Resumen
% Versión:      3.1.3 (17/04/2017)
% Codificación: UTF-8
%
% Autor: Pablo Pizarro R.
%        Facultad de Ciencias Físicas y Matemáticas.
%        Universidad de Chile.
%        pablo.pizarro@ing.uchile.cl, ppizarror.com
%
% Sitio web del proyecto: [http://ppizarror.com/Template-Informe/]
% Licencia: MIT           [https://opensource.org/licenses/MIT]

\newtitleanumheadless{Resumen}

Este informe trata sobre el desarrollo de un juego arcade de accion y plataformas. Se usaron funciones de colisión incluidas en pygame y se implementaron distintas funciones para colisionar con el nivel. Se generan además power-ups al azar y enemigos con una IA básica. Los objetivos principales fueron implementados con una serie de clases que permiten la extensión a otras estructuras de niveles y nuevos tipos de enemigos o power-ups. Se usan librerías de Python, pygame y shapely. Leer \verb!README.md! para instrucciones de uso e instalación. % Se incluye un ejemplo de resumen, se puede borrar

% TABLA DE CONTENIDOS - ÍNDICE
\input{lib/index}

% CONFIGURACIONES FINALES - INICIO DE LAS SECCIONES
\input{lib/finalconf}

% ======================== INICIO DEL DOCUMENTO ========================
\section{Introducción}
	El problema planteado era generar un juego de acción y plataformas en dos dimensiones. El juego debía permitir al usuario controlar a un personaje con total libertad para moverse dentro de la pantalla. Este personaje debía ser capaz de moverse hacia los lados, saltar y atacar. En la escena debían aparecer enemigos al azar, los cuales tenían que ser capaces de seguir al jugador, sin importar su posisción en el nivel. Un sistema para encontrar colisiones era necesario. Al hacer y recibir daño se tenían que reproducir sonidos, al igual que al saltar.

% FIN DEL DOCUMENTO
\end{document}